% Options for packages loaded elsewhere
% Options for packages loaded elsewhere
\PassOptionsToPackage{unicode}{hyperref}
\PassOptionsToPackage{hyphens}{url}
%
\documentclass[
  ignorenonframetext,
  aspectratio=169,
  russian,
]{beamer}
\newif\ifbibliography
\usepackage{pgfpages}
\setbeamertemplate{caption}[numbered]
\setbeamertemplate{caption label separator}{: }
\setbeamercolor{caption name}{fg=normal text.fg}
\beamertemplatenavigationsymbolshorizontal
% Prevent slide breaks in the middle of a paragraph
\widowpenalties 1 10000
\raggedbottom
\AtBeginPart{
  \frame{\partpage}
}
\AtBeginSection{
  \ifbibliography
  \else
    \frame{\sectionpage}
  \fi
}
\AtBeginSubsection{
  \frame{\subsectionpage}
}
\usepackage{iftex}
\ifPDFTeX
  \usepackage[T1]{fontenc}
  \usepackage[utf8]{inputenc}
  \usepackage{textcomp} % provide euro and other symbols
\else % if luatex or xetex
  \usepackage{unicode-math} % this also loads fontspec
  \defaultfontfeatures{Scale=MatchLowercase}
  \defaultfontfeatures[\rmfamily]{Ligatures=TeX,Scale=1}
\fi
\usepackage{lmodern}

\ifPDFTeX\else
  % xetex/luatex font selection
\fi
% Use upquote if available, for straight quotes in verbatim environments
\IfFileExists{upquote.sty}{\usepackage{upquote}}{}
\IfFileExists{microtype.sty}{% use microtype if available
  \usepackage[]{microtype}
  \UseMicrotypeSet[protrusion]{basicmath} % disable protrusion for tt fonts
}{}

\usepackage{color}
\usepackage{fancyvrb}
\newcommand{\VerbBar}{|}
\newcommand{\VERB}{\Verb[commandchars=\\\{\}]}
\DefineVerbatimEnvironment{Highlighting}{Verbatim}{commandchars=\\\{\}}
% Add ',fontsize=\small' for more characters per line
\usepackage{framed}
\definecolor{shadecolor}{RGB}{241,243,245}
\newenvironment{Shaded}{\begin{snugshade}}{\end{snugshade}}
\newcommand{\AlertTok}[1]{\textcolor[rgb]{0.68,0.00,0.00}{#1}}
\newcommand{\AnnotationTok}[1]{\textcolor[rgb]{0.37,0.37,0.37}{#1}}
\newcommand{\AttributeTok}[1]{\textcolor[rgb]{0.40,0.45,0.13}{#1}}
\newcommand{\BaseNTok}[1]{\textcolor[rgb]{0.68,0.00,0.00}{#1}}
\newcommand{\BuiltInTok}[1]{\textcolor[rgb]{0.00,0.23,0.31}{#1}}
\newcommand{\CharTok}[1]{\textcolor[rgb]{0.13,0.47,0.30}{#1}}
\newcommand{\CommentTok}[1]{\textcolor[rgb]{0.37,0.37,0.37}{#1}}
\newcommand{\CommentVarTok}[1]{\textcolor[rgb]{0.37,0.37,0.37}{\textit{#1}}}
\newcommand{\ConstantTok}[1]{\textcolor[rgb]{0.56,0.35,0.01}{#1}}
\newcommand{\ControlFlowTok}[1]{\textcolor[rgb]{0.00,0.23,0.31}{\textbf{#1}}}
\newcommand{\DataTypeTok}[1]{\textcolor[rgb]{0.68,0.00,0.00}{#1}}
\newcommand{\DecValTok}[1]{\textcolor[rgb]{0.68,0.00,0.00}{#1}}
\newcommand{\DocumentationTok}[1]{\textcolor[rgb]{0.37,0.37,0.37}{\textit{#1}}}
\newcommand{\ErrorTok}[1]{\textcolor[rgb]{0.68,0.00,0.00}{#1}}
\newcommand{\ExtensionTok}[1]{\textcolor[rgb]{0.00,0.23,0.31}{#1}}
\newcommand{\FloatTok}[1]{\textcolor[rgb]{0.68,0.00,0.00}{#1}}
\newcommand{\FunctionTok}[1]{\textcolor[rgb]{0.28,0.35,0.67}{#1}}
\newcommand{\ImportTok}[1]{\textcolor[rgb]{0.00,0.46,0.62}{#1}}
\newcommand{\InformationTok}[1]{\textcolor[rgb]{0.37,0.37,0.37}{#1}}
\newcommand{\KeywordTok}[1]{\textcolor[rgb]{0.00,0.23,0.31}{\textbf{#1}}}
\newcommand{\NormalTok}[1]{\textcolor[rgb]{0.00,0.23,0.31}{#1}}
\newcommand{\OperatorTok}[1]{\textcolor[rgb]{0.37,0.37,0.37}{#1}}
\newcommand{\OtherTok}[1]{\textcolor[rgb]{0.00,0.23,0.31}{#1}}
\newcommand{\PreprocessorTok}[1]{\textcolor[rgb]{0.68,0.00,0.00}{#1}}
\newcommand{\RegionMarkerTok}[1]{\textcolor[rgb]{0.00,0.23,0.31}{#1}}
\newcommand{\SpecialCharTok}[1]{\textcolor[rgb]{0.37,0.37,0.37}{#1}}
\newcommand{\SpecialStringTok}[1]{\textcolor[rgb]{0.13,0.47,0.30}{#1}}
\newcommand{\StringTok}[1]{\textcolor[rgb]{0.13,0.47,0.30}{#1}}
\newcommand{\VariableTok}[1]{\textcolor[rgb]{0.07,0.07,0.07}{#1}}
\newcommand{\VerbatimStringTok}[1]{\textcolor[rgb]{0.13,0.47,0.30}{#1}}
\newcommand{\WarningTok}[1]{\textcolor[rgb]{0.37,0.37,0.37}{\textit{#1}}}

\usepackage{longtable,booktabs,array}
\usepackage{calc} % for calculating minipage widths
\usepackage{caption}
% Make caption package work with longtable
\makeatletter
\def\fnum@table{\tablename~\thetable}
\makeatother
\usepackage{graphicx}
\makeatletter
\newsavebox\pandoc@box
\newcommand*\pandocbounded[1]{% scales image to fit in text height/width
  \sbox\pandoc@box{#1}%
  \Gscale@div\@tempa{\textheight}{\dimexpr\ht\pandoc@box+\dp\pandoc@box\relax}%
  \Gscale@div\@tempb{\linewidth}{\wd\pandoc@box}%
  \ifdim\@tempb\p@<\@tempa\p@\let\@tempa\@tempb\fi% select the smaller of both
  \ifdim\@tempa\p@<\p@\scalebox{\@tempa}{\usebox\pandoc@box}%
  \else\usebox{\pandoc@box}%
  \fi%
}
% Set default figure placement to htbp
\def\fps@figure{htbp}
\makeatother



\ifLuaTeX
\usepackage[bidi=basic,provide=*]{babel}
\else
\usepackage[bidi=default,provide=*]{babel}
\fi
% get rid of language-specific shorthands (see #6817):
\let\LanguageShortHands\languageshorthands
\def\languageshorthands#1{}


\setlength{\emergencystretch}{3em} % prevent overfull lines

\providecommand{\tightlist}{%
  \setlength{\itemsep}{0pt}\setlength{\parskip}{0pt}}



 

\usepackage[]{csquotes}

\IfFileExists{plex-otf.sty}{
  %% Full TeXlive
  % \usepackage[%
  %   % math,
  %   RM={Scale=0.94},SS={Scale=0.94},SScon={Scale=0.94},TT={Scale=MatchLowercase,FakeStretch=0.9},DefaultFeatures={Ligatures=Common}
  % ]{plex-otf}
}{
  %% TinyTeX
  \usepackage{libertine}
}

%%% Load theme
% https://deic.uab.cat/~iblanes/beamer_gallery/
\IfFileExists{beamerthemegotham.sty}{
  %% Full TeXlive
  \usetheme{gotham}
  \gothamset{
    numbering=totalpagenumber,
    parttocframe default=off,
    sectiontocframe default=off,
    subsectiontocframe default=off,
  }
}{
  %% TinyTeX
  \usetheme{Madrid}
}
\makeatletter
\@ifpackageloaded{caption}{}{\usepackage{caption}}
\AtBeginDocument{%
\ifdefined\contentsname
  \renewcommand*\contentsname{Содержание}
\else
  \newcommand\contentsname{Содержание}
\fi
\ifdefined\listfigurename
  \renewcommand*\listfigurename{Список иллюстраций}
\else
  \newcommand\listfigurename{Список иллюстраций}
\fi
\ifdefined\listtablename
  \renewcommand*\listtablename{Список таблиц}
\else
  \newcommand\listtablename{Список таблиц}
\fi
\ifdefined\figurename
  \renewcommand*\figurename{Рисунок}
\else
  \newcommand\figurename{Рисунок}
\fi
\ifdefined\tablename
  \renewcommand*\tablename{Таблица}
\else
  \newcommand\tablename{Таблица}
\fi
}
\@ifpackageloaded{float}{}{\usepackage{float}}
\floatstyle{ruled}
\@ifundefined{c@chapter}{\newfloat{codelisting}{h}{lop}}{\newfloat{codelisting}{h}{lop}[chapter]}
\floatname{codelisting}{Список}
\newcommand*\listoflistings{\listof{codelisting}{Листинги}}
\makeatother
\makeatletter
\makeatother
\makeatletter
\@ifpackageloaded{caption}{}{\usepackage{caption}}
\@ifpackageloaded{subcaption}{}{\usepackage{subcaption}}
\makeatother

\usepackage{bookmark}
\IfFileExists{xurl.sty}{\usepackage{xurl}}{} % add URL line breaks if available
\urlstyle{same}
\hypersetup{
  pdftitle={Презентация по лабораторной работе №1},
  pdfauthor={Люпп Софья Романовна},
  pdflang={ru-RU},
  hidelinks,
  pdfcreator={LaTeX via pandoc}}


\title{Презентация по лабораторной работе №1}
\subtitle{Лабораторная работа №1}
\author{Люпп Софья Романовна}
\date{2026-02-19}

\begin{document}
\frame{\titlepage}

\renewcommand*\contentsname{Содержание}
\begin{frame}[allowframebreaks]
  \frametitle{Содержание}
  \setcounter{tocdepth}{2}
  \tableofcontents
\end{frame}
\setcounter{tocdepth}{2}
\tableofcontents
}

\section{1. Информация}\label{ux438ux43dux444ux43eux440ux43cux430ux446ux438ux44f}

\begin{frame}{1.1 Докладчик}
\phantomsection\label{ux434ux43eux43aux43bux430ux434ux447ux438ux43a}
\begin{columns}[c]
\begin{column}{0.7\linewidth}
\begin{itemize}[<+->]
\tightlist
\item
  Люпп Софья Романовна
\item
  студентка группы НКНбд-01-23
\item
  кафедра теории вероятностей и кибербезопасности
\item
  Российский университет дружбы народов им. П. Лумумбы
\item
  {[}1132236039@rudn.ru{]}
\item
  \url{https://github.com/srluipp/ru/}
\end{itemize}
\end{column}

\begin{column}{0.3\linewidth}
\pandocbounded{\includegraphics[keepaspectratio]{./image/kulyabov.jpg}}
\end{column}
\end{columns}
\end{frame}

\section{2. Вводная
часть}\label{ux432ux432ux43eux434ux43dux430ux44f-ux447ux430ux441ux442ux44c}

\begin{frame}{2.1 Актуальность}
\phantomsection\label{ux430ux43aux442ux443ux430ux43bux44cux43dux43eux441ux442ux44c}
Работа с репозиториями GitHub, и, соответственно, установка таких
ресурсов, как Git-flow, работа на языке программирования Julia и др.
являются ключевыми для работы с операционными системами, командной
разработки и т.д.
\end{frame}

\begin{frame}{2.2 Объект и предмет исследования}
\phantomsection\label{ux43eux431ux44aux435ux43aux442-ux438-ux43fux440ux435ux434ux43cux435ux442-ux438ux441ux441ux43bux435ux434ux43eux432ux430ux43dux438ux44f}
Объектом и предметом исследования служат работа с терминалом Linux,
работа с системой контроля версий Git, язык программирования для
математического моделирования Julia.
\end{frame}

\begin{frame}{2.3 Цели и задачи}
\phantomsection\label{ux446ux435ux43bux438-ux438-ux437ux430ux434ux430ux447ux438}
Цель работы: познакомиться с высокоуровневым языком программирования
Julia. Задачи: ознакомиться с теорией по лабораторной работе, выполнить
задания лабораторной работы, выполнить домашние задачи контрольной
работы.
\end{frame}

\begin{frame}{2.4 Материалы и методы}
\phantomsection\label{ux43cux430ux442ux435ux440ux438ux430ux43bux44b-ux438-ux43cux435ux442ux43eux434ux44b}
Материалы: Математическое моделирование. Практикум А. В. Королькова
Кулябов Д. С.
\end{frame}

\section{3. Создание
презентации}\label{ux441ux43eux437ux434ux430ux43dux438ux435-ux43fux440ux435ux437ux435ux43dux442ux430ux446ux438ux438}

\begin{frame}{3.1 Процессор \texttt{pandoc}}
\phantomsection\label{ux43fux440ux43eux446ux435ux441ux441ux43eux440-pandoc}
\begin{itemize}[<+->]
\tightlist
\item
  Pandoc: преобразователь текстовых файлов
\item
  Сайт: \url{https://pandoc.org/}
\item
  Репозиторий: \url{https://github.com/jgm/pandoc}
\end{itemize}
\end{frame}

\begin{frame}[fragile]{3.2 Формат \texttt{pdf}}
\phantomsection\label{ux444ux43eux440ux43cux430ux442-pdf}
\begin{itemize}[<+->]
\tightlist
\item
  Использование LaTeX
\item
  Пакет для презентации: \href{https://ctan.org/pkg/beamer}{beamer}
\item
  Тема оформления: \texttt{metropolis}
\end{itemize}
\end{frame}

\begin{frame}[fragile]{3.3 Код для формата \texttt{pdf}}
\phantomsection\label{ux43aux43eux434-ux434ux43bux44f-ux444ux43eux440ux43cux430ux442ux430-pdf}
\begin{Shaded}
\begin{Highlighting}[]
\FunctionTok{slide\_level}\KeywordTok{:}\AttributeTok{ }\DecValTok{2}
\FunctionTok{aspectratio}\KeywordTok{:}\AttributeTok{ }\DecValTok{169}
\FunctionTok{section{-}titles}\KeywordTok{:}\AttributeTok{ }\CharTok{true}
\FunctionTok{theme}\KeywordTok{:}\AttributeTok{ metropolis}
\end{Highlighting}
\end{Shaded}
\end{frame}

\begin{frame}[fragile]{3.4 Формат \texttt{html}}
\phantomsection\label{ux444ux43eux440ux43cux430ux442-html}
\begin{itemize}[<+->]
\tightlist
\item
  Используется фреймворк \href{https://revealjs.com/}{reveal.js}
\item
  Используется \href{https://revealjs.com/themes/}{тема} \texttt{beige}
\end{itemize}
\end{frame}

\begin{frame}[fragile]{3.5 Код для формата \texttt{html}}
\phantomsection\label{ux43aux43eux434-ux434ux43bux44f-ux444ux43eux440ux43cux430ux442ux430-html}
\begin{itemize}[<+->]
\tightlist
\item
  Тема задаётся в файле \texttt{Makefile}
\end{itemize}

\begin{Shaded}
\begin{Highlighting}[]
\NormalTok{REVEALJS\_THEME = beige}
\end{Highlighting}
\end{Shaded}
\end{frame}

\section{4. Результаты}\label{ux440ux435ux437ux443ux43bux44cux442ux430ux442ux44b}

\begin{frame}[fragile]{4.1 Получающиеся форматы}
\phantomsection\label{ux43fux43eux43bux443ux447ux430ux44eux449ux438ux435ux441ux44f-ux444ux43eux440ux43cux430ux442ux44b}
\begin{itemize}[<+->]
\tightlist
\item
  Полученный \texttt{pdf}-файл можно демонстрировать в любой программе
  просмотра \texttt{pdf}
\item
  Полученный \texttt{html}-файл содержит в себе все ресурсы:
  изображения, css, скрипты
\end{itemize}
\end{frame}

\section{5. Элементы
презентации}\label{ux44dux43bux435ux43cux435ux43dux442ux44b-ux43fux440ux435ux437ux435ux43dux442ux430ux446ux438ux438}

\begin{frame}{5.1 Актуальность}
\phantomsection\label{ux430ux43aux442ux443ux430ux43bux44cux43dux43eux441ux442ux44c-1}
\begin{itemize}[<+->]
\tightlist
\item
  Даёт понять, о чём пойдёт речь
\item
  Следует широко и кратко описать проблему
\item
  Мотивировать свое исследование
\item
  Сформулировать цели и задачи
\item
  Возможна формулировка ожидаемых результатов
\end{itemize}
\end{frame}

\begin{frame}{5.2 Цели и задачи}
\phantomsection\label{ux446ux435ux43bux438-ux438-ux437ux430ux434ux430ux447ux438-1}
\begin{itemize}[<+->]
\tightlist
\item
  Не формулируйте более 1--2 целей исследования
\end{itemize}
\end{frame}

\begin{frame}{5.3 Материалы и методы}
\phantomsection\label{ux43cux430ux442ux435ux440ux438ux430ux43bux44b-ux438-ux43cux435ux442ux43eux434ux44b-1}
\begin{itemize}[<+->]
\tightlist
\item
  Представляйте данные качественно
\item
  Количественно, только если крайне необходимо
\item
  Излишние детали не нужны
\end{itemize}
\end{frame}

\begin{frame}{5.4 Содержание исследования}
\phantomsection\label{ux441ux43eux434ux435ux440ux436ux430ux43dux438ux435-ux438ux441ux441ux43bux435ux434ux43eux432ux430ux43dux438ux44f}
\begin{itemize}[<+->]
\tightlist
\item
  Предлагаемое решение задач исследования с обоснованием
\item
  Основные этапы работы
\end{itemize}
\end{frame}

\begin{frame}{5.5 Результаты}
\phantomsection\label{ux440ux435ux437ux443ux43bux44cux442ux430ux442ux44b-1}
\begin{itemize}[<+->]
\tightlist
\item
  Не нужны все результаты
\item
  Необходимы логические связки между слайдами
\item
  Необходимо показать понимание материала
\end{itemize}
\end{frame}

\begin{frame}{5.6 Итоговый слайд}
\phantomsection\label{ux438ux442ux43eux433ux43eux432ux44bux439-ux441ux43bux430ux439ux434}
\begin{itemize}[<+->]
\tightlist
\item
  Запоминается последняя фраза. © Штирлиц
\item
  Главное сообщение, которое вы хотите донести до слушателей
\item
  Избегайте использовать последний слайд вида \emph{Спасибо за внимание}
\end{itemize}
\end{frame}

\section{6. Рекомендую}\label{ux440ux435ux43aux43eux43cux435ux43dux434ux443ux44e}

\begin{frame}{6.1 Принцип 10/20/30}
\phantomsection\label{ux43fux440ux438ux43dux446ux438ux43f-102030}
\begin{itemize}[<+->]
\tightlist
\item
  10 слайдов
\item
  20 минут на доклад
\item
  30 кегль шрифта
\end{itemize}
\end{frame}

\begin{frame}{6.2 Связь слайдов}
\phantomsection\label{ux441ux432ux44fux437ux44c-ux441ux43bux430ux439ux434ux43eux432}
\begin{itemize}[<+->]
\tightlist
\item
  Один слайд --- одна мысль
\item
  Нельзя ссылаться на объекты, находящиеся на предыдущих слайдах
  (например, на формулы)
\item
  Каждый слайд должен иметь заголовок
\end{itemize}
\end{frame}

\begin{frame}{6.3 Количество сущностей}
\phantomsection\label{ux43aux43eux43bux438ux447ux435ux441ux442ux432ux43e-ux441ux443ux449ux43dux43eux441ux442ux435ux439}
\begin{itemize}[<+->]
\tightlist
\item
  Человек может одновременно помнить \(7 \pm 2\) элемента
\item
  При размещении информации на слайде старайтесь чтобы в сумме слайд
  содержал не более 5 элементов
\item
  Можно группировать элементы так, чтобы визуально было не более 5 групп
\end{itemize}
\end{frame}

\begin{frame}{6.4 Общие рекомендации}
\phantomsection\label{ux43eux431ux449ux438ux435-ux440ux435ux43aux43eux43cux435ux43dux434ux430ux446ux438ux438}
\begin{itemize}[<+->]
\tightlist
\item
  На слайд выносится та информация, которая без зрительной опоры
  воспринимается хуже
\item
  Слайды должны дополнять или обобщать содержание выступления или его
  частей, а не дублировать его
\item
  Информация на слайдах должна быть изложена кратко, чётко и хорошо
  структурирована
\item
  Слайд не должен быть перегружен графическими изображениями и текстом
\item
  Не злоупотребляйте анимацией и переходами
\end{itemize}
\end{frame}

\begin{frame}{6.5 Представление данных}
\phantomsection\label{ux43fux440ux435ux434ux441ux442ux430ux432ux43bux435ux43dux438ux435-ux434ux430ux43dux43dux44bux445}
\begin{itemize}[<+->]
\tightlist
\item
  Лучше представить в виде схемы
\item
  Менее оптимально представить в виде рисунка, графика, таблицы
\item
  Текст используется, если все предыдущие способы отображения информации
  не подошли
\end{itemize}
\end{frame}




\end{document}
