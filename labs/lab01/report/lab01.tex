% Options for packages loaded elsewhere
% Options for packages loaded elsewhere
\PassOptionsToPackage{unicode}{hyperref}
\PassOptionsToPackage{hyphens}{url}
%
\documentclass[
  english,
  russian,
  12pt,
  a4paper,
  DIV=11,
  numbers=noendperiod]{scrreprt}
\usepackage{xcolor}
\usepackage{amsmath,amssymb}
\setcounter{secnumdepth}{5}
\usepackage{iftex}
\ifPDFTeX
  \usepackage[T1]{fontenc}
  \usepackage[utf8]{inputenc}
  \usepackage{textcomp} % provide euro and other symbols
\else % if luatex or xetex
  \usepackage{unicode-math} % this also loads fontspec
  \defaultfontfeatures{Scale=MatchLowercase}
  \defaultfontfeatures[\rmfamily]{Ligatures=TeX,Scale=1}
\fi
\usepackage{lmodern}
\ifPDFTeX\else
  % xetex/luatex font selection
\fi
% Use upquote if available, for straight quotes in verbatim environments
\IfFileExists{upquote.sty}{\usepackage{upquote}}{}
\IfFileExists{microtype.sty}{% use microtype if available
  \usepackage[]{microtype}
  \UseMicrotypeSet[protrusion]{basicmath} % disable protrusion for tt fonts
}{}
\usepackage{setspace}
% Make \paragraph and \subparagraph free-standing
\makeatletter
\ifx\paragraph\undefined\else
  \let\oldparagraph\paragraph
  \renewcommand{\paragraph}{
    \@ifstar
      \xxxParagraphStar
      \xxxParagraphNoStar
  }
  \newcommand{\xxxParagraphStar}[1]{\oldparagraph*{#1}\mbox{}}
  \newcommand{\xxxParagraphNoStar}[1]{\oldparagraph{#1}\mbox{}}
\fi
\ifx\subparagraph\undefined\else
  \let\oldsubparagraph\subparagraph
  \renewcommand{\subparagraph}{
    \@ifstar
      \xxxSubParagraphStar
      \xxxSubParagraphNoStar
  }
  \newcommand{\xxxSubParagraphStar}[1]{\oldsubparagraph*{#1}\mbox{}}
  \newcommand{\xxxSubParagraphNoStar}[1]{\oldsubparagraph{#1}\mbox{}}
\fi
\makeatother


\usepackage{longtable,booktabs,array}
\usepackage{calc} % for calculating minipage widths
% Correct order of tables after \paragraph or \subparagraph
\usepackage{etoolbox}
\makeatletter
\patchcmd\longtable{\par}{\if@noskipsec\mbox{}\fi\par}{}{}
\makeatother
% Allow footnotes in longtable head/foot
\IfFileExists{footnotehyper.sty}{\usepackage{footnotehyper}}{\usepackage{footnote}}
\makesavenoteenv{longtable}
\usepackage{graphicx}
\makeatletter
\newsavebox\pandoc@box
\newcommand*\pandocbounded[1]{% scales image to fit in text height/width
  \sbox\pandoc@box{#1}%
  \Gscale@div\@tempa{\textheight}{\dimexpr\ht\pandoc@box+\dp\pandoc@box\relax}%
  \Gscale@div\@tempb{\linewidth}{\wd\pandoc@box}%
  \ifdim\@tempb\p@<\@tempa\p@\let\@tempa\@tempb\fi% select the smaller of both
  \ifdim\@tempa\p@<\p@\scalebox{\@tempa}{\usebox\pandoc@box}%
  \else\usebox{\pandoc@box}%
  \fi%
}
% Set default figure placement to htbp
\def\fps@figure{htbp}
\makeatother



\ifLuaTeX
\usepackage[bidi=basic,provide=*]{babel}
\else
\usepackage[bidi=default,provide=*]{babel}
\fi
% get rid of language-specific shorthands (see #6817):
\let\LanguageShortHands\languageshorthands
\def\languageshorthands#1{}


\setlength{\emergencystretch}{3em} % prevent overfull lines

\providecommand{\tightlist}{%
  \setlength{\itemsep}{0pt}\setlength{\parskip}{0pt}}



 
\usepackage[style=gost-numeric,backend=biber,langhook=extras,autolang=other*]{biblatex}
\addbibresource{bib/cite.bib}

\usepackage[]{csquotes}

\usepackage{indentfirst}
\usepackage{float}
\floatplacement{figure}{H}
\IfFileExists{plex-otf.sty}{
  %% Full TeXlive
  \usepackage[
    % math,
    RM={Scale=0.94},SS={Scale=0.94},SScon={Scale=0.94},TT={Scale=MatchLowercase,FakeStretch=0.9},
    DefaultFeatures={Ligatures=Common}
  ]{plex-otf}
  % \usepackage[Scale=MatchUppercase]{juliamono}
  \setmonofont{JuliaMono}[Scale=0.9, Ligatures=Common] 
}{
  %% TinyTeX
  \usepackage{libertine}
  \usepackage{fontspec} 
  \setmonofont{JuliaMono}[Scale=0.9, Ligatures=Common]
}
\KOMAoption{captions}{tableheading}
\makeatletter
\@ifpackageloaded{caption}{}{\usepackage{caption}}
\AtBeginDocument{%
\ifdefined\contentsname
  \renewcommand*\contentsname{Содержание}
\else
  \newcommand\contentsname{Содержание}
\fi
\ifdefined\listfigurename
  \renewcommand*\listfigurename{Список иллюстраций}
\else
  \newcommand\listfigurename{Список иллюстраций}
\fi
\ifdefined\listtablename
  \renewcommand*\listtablename{Список таблиц}
\else
  \newcommand\listtablename{Список таблиц}
\fi
\ifdefined\figurename
  \renewcommand*\figurename{Рисунок}
\else
  \newcommand\figurename{Рисунок}
\fi
\ifdefined\tablename
  \renewcommand*\tablename{Таблица}
\else
  \newcommand\tablename{Таблица}
\fi
}
\@ifpackageloaded{float}{}{\usepackage{float}}
\floatstyle{ruled}
\@ifundefined{c@chapter}{\newfloat{codelisting}{h}{lop}}{\newfloat{codelisting}{h}{lop}[chapter]}
\floatname{codelisting}{Список}
\newcommand*\listoflistings{\listof{codelisting}{Листинги}}
\makeatother
\makeatletter
\makeatother
\makeatletter
\@ifpackageloaded{caption}{}{\usepackage{caption}}
\@ifpackageloaded{subcaption}{}{\usepackage{subcaption}}
\makeatother
\usepackage{bookmark}
\IfFileExists{xurl.sty}{\usepackage{xurl}}{} % add URL line breaks if available
\urlstyle{same}
\hypersetup{
  pdftitle={Лабораторная работа №1},
  pdfauthor={Люпп Софья Романовна},
  pdflang={ru-RU},
  hidelinks,
  pdfcreator={LaTeX via pandoc}}


\title{Лабораторная работа №1}
\author{Люпп Софья Романовна}
\date{}
\begin{document}
\maketitle

\renewcommand*\contentsname{Содержание}
{
\setcounter{tocdepth}{1}
\tableofcontents
}
\listoffigures
\listoftables

\setstretch{1.5}
\chapter{Цель
работы}\label{ux446ux435ux43bux44c-ux440ux430ux431ux43eux442ux44b}

Цель данной лабораторной работы - приобретение практических навыков
работы с системой управления версиями Git, языком программирования
Julia.

\chapter{Задание}\label{ux437ux430ux434ux430ux43dux438ux435}

\begin{enumerate}
\def\labelenumi{\arabic{enumi}.}
\item
  Выполнить задания из лабораторной работы
\item
  Оформить отчет по лабораторной работе с помощью Markdown
\item
\begin{verbatim}
 Выполнить домашнюю работу
\end{verbatim}
\end{enumerate}

\chapter{Теоретическое
введение}\label{ux442ux435ux43eux440ux435ux442ux438ux447ux435ux441ux43aux43eux435-ux432ux432ux435ux434ux435ux43dux438ux435}

Системы контроля версий (Version Control System, VCS) применяются при
работе нескольких человек над одним проектом. Обычно основное дерево
проекта хранится в локальном или удалённом репозитории, к которому
настроен доступ для участников проекта. При внесении изменений в
содержание проекта система контроля версий позволяет их фиксировать,
совмещать изменения, произведённые разными участниками проекта,
производить откат к любой более ранней версии проекта, если это
требуется.

Системы контроля версий также могут обеспечивать дополнительные, более
гибкие функциональные возможности. Например, они могут поддерживать
работу с несколькими версиями одного файла, сохраняя общую историю
изменений до точки ветвления версий и собственные истории изменений
каждой ветви. Кроме того, обычно доступна информация о том, кто из
участников, когда и какие изменения вносил. Обычно такого рода
информация хранится в журнале изменений, доступ к которому можно
ограничить.

1.2.1.4 Создание ключа ssh 1. Общая информация 1. Алгоритмы шифрования
ssh 1. Аутентификация В SSH поддерживается четыре алгоритма
аутентификации по открытым ключам: --- DSA: --- размер ключей DSA не
может превышать 1024, его следует отключить; --- RSA: --- следует
создавать ключ большого размера: 4096 бит; --- ECDSA: --- ECDSA завязан
на технологиях NIST, его следует отключить; --- Ed25519: ---
используется пока не везде.

По умолчанию пользовательские ssh-ключи сохраняются в каталоге
\textasciitilde/.ssh в домашнем каталоге пользователя.

Создание ключа ssh --- Ключ ssh создаётся командой: ssh-keygen -t

1.2.2.5.1 Основные ветки (master) и ветки разработки (develop) Для
фиксации истории проекта в рамках этого процесса вместо одной ветки
master используются две ветки. В ветке master хранится официальная
история релиза, а ветка develop предназначена для объединения всех
функций. Кроме того, для удобства рекомендуется присваивать всем
коммитам в ветке master номер версии.

1.2.2.5.2 Функциональные ветки (feature)

Под каждую новую функцию должна быть отведена собственная ветка, которую
можно отправлять в центральный репозиторий для создания резервной копии
или совместной работы команды. Ветки feature создаются не на основе
master, а на основе develop. Когда работа над функцией завершается,
соответствующая ветка сливается обратно с веткой develop. Функции не
следует отправлять напрямую в ветку master. Как правило, ветки feature
создаются на основе последней ветки develop.

2.1 Литературное (грамотное) программирование --- это подход,
приоритизирующий понятность программы для человека, а не её исполнение
компьютером. В экосистеме Julia он реализуется через несколько
инструментов. Предлагается использовать Literate.jl.

Идея, предложенная Дональдом Кнутом {[}1---3{]}, меняет традиционный
взгляд: программа создаётся не как последовательность инструкций для
машины, а как литературное эссе, объясняющее логику решения задачи.

--- Literate.jl отлично подходит для автоматизации создания
документации, а также для написания статей и учебников прямо в
.jl-файлах. --- Weave.jl больше заточен под академические отчёты и
публикации, требующие строгого форматирования (LaTeX/PDF). --- Jupyter
лучше подходит для интерактивного исследования данных, но управление
версиями .ipynb-файлов и их автоматизация сложнее.

\chapter{Выполнение лабораторной
работы}\label{ux432ux44bux43fux43eux43bux43dux435ux43dux438ux435-ux43bux430ux431ux43eux440ux430ux442ux43eux440ux43dux43eux439-ux440ux430ux431ux43eux442ux44b}

Начинаю с подготовки и настройки рабочего пространства (установка git)
(рис.~\ref{fig-001}).

\begin{figure}

\centering{

\includegraphics[width=0.7\linewidth,height=\textheight,keepaspectratio]{image/1.png}

}

\caption{\label{fig-001}git1}

\end{figure}%

Настраиваю git configs(рис.~\ref{fig-002}).

\begin{figure}

\centering{

\includegraphics[width=0.7\linewidth,height=\textheight,keepaspectratio]{image/2.png}

}

\caption{\label{fig-002}gitconfigs}

\end{figure}%

Пробую делать коммиты(рис.~\ref{fig-003}).

\begin{figure}

\centering{

\includegraphics[width=0.7\linewidth,height=\textheight,keepaspectratio]{image/3.png}

}

\caption{\label{fig-003}commits}

\end{figure}%

Настраиваю gitignores(рис.~\ref{fig-004}) (рис.~\ref{fig-005}).

\begin{figure}

\centering{

\includegraphics[width=0.7\linewidth,height=\textheight,keepaspectratio]{image/4.png}

}

\caption{\label{fig-004}gitignores}

\end{figure}%

\begin{figure}

\centering{

\includegraphics[width=0.7\linewidth,height=\textheight,keepaspectratio]{image/5.png}

}

\caption{\label{fig-005}gitignores2}

\end{figure}%

Далее работаю с настройкой ключей ssh-keygen (рис.~\ref{fig-006}).

\begin{figure}

\centering{

\includegraphics[width=0.7\linewidth,height=\textheight,keepaspectratio]{image/6.png}

}

\caption{\label{fig-006}ssh}

\end{figure}%

Получаю ключ rsa (рис.~\ref{fig-007}).

\begin{figure}

\centering{

\includegraphics[width=0.7\linewidth,height=\textheight,keepaspectratio]{image/7.png}

}

\caption{\label{fig-007}rsa}

\end{figure}%

Работаю с ветками git master, git origin (рис.~\ref{fig-008}).

\begin{figure}

\centering{

\includegraphics[width=0.7\linewidth,height=\textheight,keepaspectratio]{image/8.png}

}

\caption{\label{fig-008}git}

\end{figure}%

Генерирую ключ gpg (рис.~\ref{fig-009}).

\begin{figure}

\centering{

\includegraphics[width=0.7\linewidth,height=\textheight,keepaspectratio]{image/9.png}

}

\caption{\label{fig-009}gpg}

\end{figure}%

Устанавливаю необходимые параметры для ключа: 4096, rsa, key does not
expire at all (0) (рис.~\ref{fig-010}) (рис.~\ref{fig-011})..

\begin{figure}

\centering{

\includegraphics[width=0.7\linewidth,height=\textheight,keepaspectratio]{image/10.png}

}

\caption{\label{fig-010}gpg1}

\end{figure}%

\begin{figure}

\centering{

\includegraphics[width=0.7\linewidth,height=\textheight,keepaspectratio]{image/11.png}

}

\caption{\label{fig-011}gpg2}

\end{figure}%

Затем копирую ключ и прикрепляю новый gpg ключ на свой профиль в GitHub
(рис.~\ref{fig-012}) (рис.~\ref{fig-013})..

\begin{figure}

\centering{

\includegraphics[width=0.7\linewidth,height=\textheight,keepaspectratio]{image/12.png}

}

\caption{\label{fig-012}gpg3}

\end{figure}%

\begin{figure}

\centering{

\includegraphics[width=0.7\linewidth,height=\textheight,keepaspectratio]{image/13.png}

}

\caption{\label{fig-013}gpg4}

\end{figure}%

Настраиваю configs (рис.~\ref{fig-014}).

\begin{figure}

\centering{

\includegraphics[width=0.7\linewidth,height=\textheight,keepaspectratio]{image/14.png}

}

\caption{\label{fig-014}configs2}

\end{figure}%

Теперь необходимо установить и распаковать gitflow (рис.~\ref{fig-015})
(рис.~\ref{fig-016}) (рис.~\ref{fig-017}).

\begin{figure}

\centering{

\includegraphics[width=0.7\linewidth,height=\textheight,keepaspectratio]{image/15.png}

}

\caption{\label{fig-015}gitflow1}

\end{figure}%

\begin{figure}

\centering{

\includegraphics[width=0.7\linewidth,height=\textheight,keepaspectratio]{image/16.png}

}

\caption{\label{fig-016}gitflow2}

\end{figure}%

\begin{figure}

\centering{

\includegraphics[width=0.7\linewidth,height=\textheight,keepaspectratio]{image/17.png}

}

\caption{\label{fig-017}gitflow3}

\end{figure}%

Настройка branches (рис.~\ref{fig-018}).

\begin{figure}

\centering{

\includegraphics[width=0.7\linewidth,height=\textheight,keepaspectratio]{image/18.png}

}

\caption{\label{fig-018}branches}

\end{figure}%

Checking the branches (рис.~\ref{fig-019}) (рис.~\ref{fig-020}).

\begin{figure}

\centering{

\includegraphics[width=0.7\linewidth,height=\textheight,keepaspectratio]{image/19.png}

}

\caption{\label{fig-019}branches2}

\end{figure}%

\begin{figure}

\centering{

\includegraphics[width=0.7\linewidth,height=\textheight,keepaspectratio]{image/20.png}

}

\caption{\label{fig-020}branches3}

\end{figure}%

Так же необходимо предустановить gh (рис.~\ref{fig-021}).

\begin{figure}

\centering{

\includegraphics[width=0.7\linewidth,height=\textheight,keepaspectratio]{image/21.png}

}

\caption{\label{fig-021}gh}

\end{figure}%

Установленный gitflow (рис.~\ref{fig-022}).

\begin{figure}

\centering{

\includegraphics[width=0.7\linewidth,height=\textheight,keepaspectratio]{image/22.png}

}

\caption{\label{fig-022}gitflow}

\end{figure}%

По заданию нужно установить nodejs (рис.~\ref{fig-023})
(рис.~\ref{fig-024}) (рис.~\ref{fig-025}).

\begin{figure}

\centering{

\includegraphics[width=0.7\linewidth,height=\textheight,keepaspectratio]{image/23.png}

}

\caption{\label{fig-023}nodejs}

\end{figure}%

\begin{figure}

\centering{

\includegraphics[width=0.7\linewidth,height=\textheight,keepaspectratio]{image/24.png}

}

\caption{\label{fig-024}nodejs2}

\end{figure}%

\begin{figure}

\centering{

\includegraphics[width=0.7\linewidth,height=\textheight,keepaspectratio]{image/25.png}

}

\caption{\label{fig-025}nodejs3}

\end{figure}%

Устанавливаю libxcrip (рис.~\ref{fig-026}).

\begin{figure}

\centering{

\includegraphics[width=0.7\linewidth,height=\textheight,keepaspectratio]{image/26.png}

}

\caption{\label{fig-026}libxcript}

\end{figure}%

Установка quarto (рис.~\ref{fig-027}).

\begin{figure}

\centering{

\includegraphics[width=0.7\linewidth,height=\textheight,keepaspectratio]{image/27.png}

}

\caption{\label{fig-027}quarto}

\end{figure}%

Добавила в преамбуле preamble.tex пакет juliamono (рис.~\ref{fig-028})
(рис.~\ref{fig-029}).

\begin{figure}

\centering{

\includegraphics[width=0.7\linewidth,height=\textheight,keepaspectratio]{image/28.png}

}

\caption{\label{fig-028}juliamono}

\end{figure}%

\begin{figure}

\centering{

\includegraphics[width=0.7\linewidth,height=\textheight,keepaspectratio]{image/29.png}

}

\caption{\label{fig-029}juliamono2}

\end{figure}%

В каталоге отчёта в файл \_quarto.yml включила поддержку кода julia.
Добавила после описания проекта (рис.~\ref{fig-030})
(рис.~\ref{fig-031}) (рис.~\ref{fig-032}).

\begin{figure}

\centering{

\includegraphics[width=0.7\linewidth,height=\textheight,keepaspectratio]{image/30.png}

}

\caption{\label{fig-030}quarto2}

\end{figure}%

\begin{figure}

\centering{

\includegraphics[width=0.7\linewidth,height=\textheight,keepaspectratio]{image/31.png}

}

\caption{\label{fig-031}quarto3}

\end{figure}%

\begin{figure}

\centering{

\includegraphics[width=0.7\linewidth,height=\textheight,keepaspectratio]{image/32.png}

}

\caption{\label{fig-032}quarto4}

\end{figure}%

Устанавливаю jupiter (рис.~\ref{fig-033}).

\begin{figure}

\centering{

\includegraphics[width=0.7\linewidth,height=\textheight,keepaspectratio]{image/33.png}

}

\caption{\label{fig-033}jupiter}

\end{figure}%

Через терминал вывожу jupiter в mozila firefox (рис.~\ref{fig-034}).

\begin{figure}

\centering{

\includegraphics[width=0.7\linewidth,height=\textheight,keepaspectratio]{image/34.png}

}

\caption{\label{fig-034}jupiter2}

\end{figure}%

Генерация файла с экспоненциальным ростом через julia
(рис.~\ref{fig-035}).

\begin{figure}

\centering{

\includegraphics[width=0.7\linewidth,height=\textheight,keepaspectratio]{image/35.png}

}

\caption{\label{fig-035}exponential}

\end{figure}%

Презентация экспоненциального роста в папке plots (рис.~\ref{fig-036}).

\begin{figure}

\centering{

\includegraphics[width=0.7\linewidth,height=\textheight,keepaspectratio]{image/36.png}

}

\caption{\label{fig-036}juliaplots}

\end{figure}%

Вывод файла exponential (рис.~\ref{fig-037}) (рис.~\ref{fig-038})
(рис.~\ref{fig-039}) (рис.~\ref{fig-040}).

\begin{figure}

\centering{

\includegraphics[width=0.7\linewidth,height=\textheight,keepaspectratio]{image/37.png}

}

\caption{\label{fig-037}result}

\end{figure}%

\begin{figure}

\centering{

\includegraphics[width=0.7\linewidth,height=\textheight,keepaspectratio]{image/38.png}

}

\caption{\label{fig-038}results}

\end{figure}%

\begin{figure}

\centering{

\includegraphics[width=0.7\linewidth,height=\textheight,keepaspectratio]{image/39.png}

}

\caption{\label{fig-039}results1}

\end{figure}%

\begin{figure}

\centering{

\includegraphics[width=0.7\linewidth,height=\textheight,keepaspectratio]{image/40.png}

}

\caption{\label{fig-040}result2}

\end{figure}%

Создаю новый файл для описания экспоненциального роста
(рис.~\ref{fig-041}) (рис.~\ref{fig-042}).

\begin{figure}

\centering{

\includegraphics[width=0.7\linewidth,height=\textheight,keepaspectratio]{image/41.png}

}

\caption{\label{fig-041}exp}

\end{figure}%

\begin{figure}

\centering{

\includegraphics[width=0.7\linewidth,height=\textheight,keepaspectratio]{image/42.png}

}

\caption{\label{fig-042}exp2}

\end{figure}%

Устанавливаю пакеты DrWatson и другие (рис.~\ref{fig-043})
(рис.~\ref{fig-044}) (рис.~\ref{fig-045})(рис.~\ref{fig-046}) .

\begin{figure}

\centering{

\includegraphics[width=0.7\linewidth,height=\textheight,keepaspectratio]{image/43.png}

}

\caption{\label{fig-043}drwatson}

\end{figure}%

\begin{figure}

\centering{

\includegraphics[width=0.7\linewidth,height=\textheight,keepaspectratio]{image/44.png}

}

\caption{\label{fig-044}packets}

\end{figure}%

\begin{figure}

\centering{

\includegraphics[width=0.7\linewidth,height=\textheight,keepaspectratio]{image/45.png}

}

\caption{\label{fig-045}packet}

\end{figure}%

\begin{figure}

\centering{

\includegraphics[width=0.7\linewidth,height=\textheight,keepaspectratio]{image/46.png}

}

\caption{\label{fig-046}doneinstall}

\end{figure}%

Работа с Julia (рис.~\ref{fig-047}).

\begin{figure}

\centering{

\includegraphics[width=0.7\linewidth,height=\textheight,keepaspectratio]{image/47.png}

}

\caption{\label{fig-047}julia1}

\end{figure}%

Установка пакетов для Julia (рис.~\ref{fig-048}) (рис.~\ref{fig-049})
(рис.~\ref{fig-050}) (рис.~\ref{fig-051}).

\begin{figure}

\centering{

\includegraphics[width=0.7\linewidth,height=\textheight,keepaspectratio]{image/48.png}

}

\caption{\label{fig-048}juliapackets}

\end{figure}%

\begin{figure}

\centering{

\includegraphics[width=0.7\linewidth,height=\textheight,keepaspectratio]{image/49.png}

}

\caption{\label{fig-049}juliapackwt}

\end{figure}%

\begin{figure}

\centering{

\includegraphics[width=0.7\linewidth,height=\textheight,keepaspectratio]{image/50.png}

}

\caption{\label{fig-050}juliapac}

\end{figure}%

\begin{figure}

\centering{

\includegraphics[width=0.7\linewidth,height=\textheight,keepaspectratio]{image/51.png}

}

\caption{\label{fig-051}error}

\end{figure}%

Знакомство с julia через терминал (рис.~\ref{fig-052}).

\begin{figure}

\centering{

\includegraphics[width=0.7\linewidth,height=\textheight,keepaspectratio]{image/52.png}

}

\caption{\label{fig-052}acquaintancejulia}

\end{figure}%

Загрузка Julia вручную (рис.~\ref{fig-053}).

\begin{figure}

\centering{

\includegraphics[width=0.7\linewidth,height=\textheight,keepaspectratio]{image/53.png}

}

\caption{\label{fig-053}installjulia}

\end{figure}%

\chapter{Выводы}\label{ux432ux44bux432ux43eux434ux44b}

В ходе лабораторной работы я приобрела практические навыки работы с
системой управления версиями Git, языком программирования Julia.

\chapter*{Список
литературы}\label{ux441ux43fux438ux441ux43eux43a-ux43bux438ux442ux435ux440ux430ux442ux443ux440ux44b}
\addcontentsline{toc}{chapter}{Список литературы}

\printbibliography[heading=none]





\end{document}
